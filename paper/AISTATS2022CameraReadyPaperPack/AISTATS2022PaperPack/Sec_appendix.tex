
\section{Entropic UOT dual}
We know that 
\begin{equation}
P(t):= \min_{t \in \mathbb{R}_{+}^{mn}} f(\X t) + g(t)
\end{equation}
the dual problem is 
\begin{equation}
D(\theta) := \max_{\theta} -f^*(-\theta) - g^*(X^{\tranT}\theta)
\end{equation}
The Entropic UOT problem is:
\begin{equation}
\label{eq:euot}
W(\alpha,\beta) := \min_{t \in \mathbb{R}_{+}^{mn}, t_i>\epsilon>0} \lambda c^{\tranT}t + D_h(\mathbf{X}t,y) + \varepsilon H(t)
\end{equation}

In order to screening, we introduce a smoothing threshold $\eta_1,\eta_2 > 0$ 
$$
\begin{aligned}
f(\X t) &=  D_h(\mathbf{X}t,y) \\
g(t) &= \lambda c^{\tranT} t + \varepsilon H(t)
\end{aligned}
$$

We have 
$$
\begin{aligned}
f^{*}(\theta) &= \max_t \theta^{\tranT} t - t\ln \frac{t}{y} - t+y\\
&=ye^{\theta} - y\\
g^{*}(\theta) &= \max_t \theta^{\tranT} t - \varepsilon t\ln t - \lambda c^{\tranT}t\\
&= \sum_{i,\theta_i\geq\lambda c_i +\varepsilon(\ln \epsilon + 1)} \varepsilon \exp(\frac{\theta_i-\lambda c_i}{\varepsilon}-1)+\sum_{i,\theta_i<\lambda c_i +\varepsilon(\ln \epsilon + 1)} \epsilon(\theta_i - \varepsilon\ln\epsilon-\lambda c_i)
\end{aligned}
$$

The dual problem is:
\begin{equation}
\label{eq:reuot}
D(\theta) := \max_{\theta} -ye^{-\theta} - y - \sum_{i,x_i^{\tranT}\theta\geq\lambda c_i +\varepsilon(\ln \epsilon + 1)} \varepsilon \exp(\frac{x_i^{\tranT}\theta-\lambda c_i}{\varepsilon}-1) - \sum_{i,x_i^{\tranT}\theta<\lambda c_i +\varepsilon(\ln \epsilon + 1)} \epsilon(x_i^{\tranT}\theta - \varepsilon\ln\epsilon-\lambda c_i)
\end{equation}

Let's have a look whether it is strongly convex...
$$
\begin{aligned}
\frac{\partial D}{\partial \theta} &=  ye^{-\theta} -  \sum_{i,x_i^{\tranT}\theta\geq\lambda c_i +\varepsilon(\ln \epsilon + 1)} x_i\exp(\frac{x_i^{\tranT}\theta-\lambda c_i}{\varepsilon}-1) -\sum_{i,x_i^{\tranT}\theta<\lambda c_i +\varepsilon(\ln \epsilon + 1)} \epsilon x_i
\end{aligned}
$$

$$
\begin{aligned}
\frac{\partial^{2} D}{\partial \theta^{2}} &=  -\operatorname{Diag}(ye^{-\theta}) -  \sum_{i,x_i^{\tranT}\theta\geq\lambda c_i +\varepsilon(\ln \epsilon + 1)} \frac{x_i x_i^{\tranT}}{\varepsilon}\exp(\frac{x_i^{\tranT}\theta-\lambda c_i}{\varepsilon}-1)
\end{aligned}
$$


$$
\begin{aligned}
L(t,v,\eta,u,m) &= \min_{t,v} \max_{\eta>0,u,m} \lambda c^{\tranT}t + D_h(y,v) + \varepsilon H(t) + \eta ^{\tranT} (-t) +u^{\tranT}(v-\X t) + m( t^{\tranT}\one - a)\\
&= \max_{\eta>0,u,m} -ma + \min_{t,v}  \lambda c^{\tranT}t + D_h(y,v) + \varepsilon H(t) + \eta ^{\tranT} (-t) +u^{\tranT}(v-\X t) + m t^{\tranT}\one 
\end{aligned}
$$

\begin{equation}
\frac{\partial L}{\partial v} = 
\end{equation}

\begin{equation}
W(\alpha,\beta) := \min_{t \in \mathbb{R}_{+}^{mn}} f(\X t) + g(t)
\end{equation}
We have 
$$
f(Xt) = D_h(y,\X t) 
$$

$$
g(t) = \lambda c^{\tranT}t +  \varepsilon (t +\eta_2) \ln (t+\eta_2)
$$

$$
\begin{aligned}
g^*(\theta) &= \max_{t} \theta^{\tranT} t - g(x)\\
\frac{\partial g^*}{\partial t} &= \theta - \lambda c - \varepsilon(\ln (t+\eta_2)+1) \\
t^* &= \exp (\frac{\theta - \lambda c}{\varepsilon}-1) - \eta_2\\
g^*(\theta) &= (\lambda c-\theta)\eta_2 + \varepsilon \exp(\frac{\lambda c-\theta}{\varepsilon}-1)
\mathbf{s.t.} 
\end{aligned}
$$

\section{FORMATTING INSTRUCTIONS FOR THE SUPPLEMENTARY MATERIAL}

Your supplementary material should go here. It may be in one-column or two-column format. To display the supplementary material in two-column format, comment out the line
\begin{verbatim}
\onecolumn \makesupplementtitle
\end{verbatim}
and uncomment the following line:
\begin{verbatim}
\twocolumn[ \makesupplementtitle ]
\end{verbatim}

Please submit your paper (including the supplementary material) as a single PDF file. Besides the PDF file, you may submit a single file of additional non-textual supplementary material, which should be a ZIP file containing, e.g., code.

If you require to upload any video as part of the supplementary material of your camera-ready submission, do not submit it in the ZIP file. Instead, please send us via email the URL containing the video location.

Note that reviewers are under no obligation to examine your supplementary material.

\section{MISSING PROOFS}

The supplementary materials may contain detailed proofs of the results that are missing in the main paper.

\subsection{Proof of Lemma 3}

\textit{In this section, we present the detailed proof of Lemma 3 and then [ ... ]}

\section{ADDITIONAL EXPERIMENTS}

If you have additional experimental results, you may include them in the supplementary materials.

\subsection{The Effect of Regularization Parameter}

\textit{Our algorithm depends on the regularization parameter $\lambda$. Here we illustrate the effect of this parameter on the performance of our algorithm [ ... ]}


