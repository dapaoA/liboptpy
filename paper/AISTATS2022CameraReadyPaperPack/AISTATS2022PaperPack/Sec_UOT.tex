\section{Dynamic Screening and UOT problem}
\subsection{Screening for UOT}

For UOT ptoblem \ref{eq:uot}, we could get its dual form. 

\begin{equation}
\begin{split}
-d^*(-\theta) - g^*(X^{\tranT}\theta) =& -\frac{1}{2}\|\theta\|_2^2-y^{\tranT}\theta \\
 \text{s.t.} &\forall i \quad x_i^{\tranT}\theta -\lambda c_i \leq 0
 \end{split}
 \label{eq:uotdual}
\end{equation}
The equation indicate that the dual problem has many dual constraints, the optimal solution is inside the constaints.\\
From the KKT condition, we can make sure that, for the optimal primal solution $\hat{t}$:
 \begin{equation}
\begin{split}
x_i^{\tranT}\theta -\lambda c_i  \left\{
\begin{aligned}
< 0 \quad& \Rightarrow \hat{t}_i = 0\\
= 0 \quad& \Rightarrow \hat{t}_i \geq 0
\end{aligned}
\right.
 \end{split}
 \label{eq:screening}
\end{equation}

As we do not know the information of $\hat{t}$ directly, we can construct an area $\mathcal{R}^{S}$ containing the $\hat{t}$, if

 \begin{equation}
\max_{t \in \mathcal{R}^S} x_i^{\tranT}\theta -\lambda c_i  < 0
\end{equation}
then we have:
 \begin{equation}
 x_i^{\tranT}\hat{\theta} -\lambda c_i  < 0
\end{equation}
which means the corresponding $\hat{t}_i = 0$, and can be screening out.

Now we start to construct the area containing $\hat{\theta}$, from \ref{circle} we know that the $\hat{\theta}$ is inside the intersection of the area of \ref{circle} and the dual feasible area. However, the multilinear constraints make it hard to compute the maximum for the problem, We design a relaxation method. which divide the constrains into two parts, then we are maximizing on the intersection of two hyperplane and a hyper-ball. 

\begin{thm}\label{area}
$$
\mathcal{R}^S = \{ nihao \}
$$
\end{thm}
the computational process is in Appendix.A

\subsection{Algorithms}

 \begin{algorithm}
 \caption{Algorithm for ...}
 \begin{algorithmic}[1]
 \renewcommand{\algorithmicrequire}{\textbf{Input:}}
 \renewcommand{\algorithmicensure}{\textbf{Output:}}
 \REQUIRE in
 \ENSURE  out
 \\ \textit{Initialisation} :
  \STATE first statement
 \\ \textit{LOOP Process}
  \FOR {$i = l-2$ to $0$}
  \STATE statements..
  \IF {($i \ne 0$)}
  \STATE statement..
  \ENDIF
  \ENDFOR
 \RETURN $P$ 
 \end{algorithmic} 
 \end{algorithm}

screening method is irrelevent to the optimization solver you choose. We gave the specific algorithm for $L_2$ UOT problem to show the whole optimization process.\\
An important part is that we use the primal solution to compute a dual solution, which might not inside the dual feasible area, a projection method is necessary. We promote a new projection method for the UOT problem for its sparse matrix structure.

\begin{thm}\label{proj}
$$
\tilde{\theta} = 1
$$
\end{thm}











































