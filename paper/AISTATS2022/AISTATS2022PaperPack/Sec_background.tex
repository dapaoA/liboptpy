\section{BACKGROUND}
\subsection{Optimal Transport and Unbalanced Optimal Transport}
Given two histograms $\ma\in \R^{m}, \mb \in \R^{n},$ For a cost matrix $\C \in \mathbbm{R_{+}}^{m \times n}$, mordern Optimal transport problem is trying to get a corresponding transport matrix $\T \in \R_{+}^{m \times n}$ that minimize the whole transport cost, which could be formulated as:
\begin{equation}
\begin{split}
&\operatorname{OT}(\ma,\mb) := \min_{ \T \in \R_{+}^{n \times n}} \langle \C, \T \rangle \\
& \mathbf{T} \one_n= \ma, \mathbf{T}^{T}\one_m = \mb
\end{split}
\end{equation}

We can write it into a vector type, set $\vc,\vt \in \mathbbm{R}^{mn}$:
\begin{equation}
\begin{split}
&\operatorname{OT}(\ma,\mb) := \min_{t \in \R_{+}^{n^2}} c^{\tranT}\vt \\
& \mathbf{N}\vt = \alpha, \mathbf{M}\vt = \beta
\end{split}
\end{equation}

$\mathbf{N} \in \R^{m \times mn}, \mathbf{N} \in \R^{n \times mn}$ are two matrix consisted with 0 and 1, listed in Appendix.A. When the $\|\ma\|_2 = \|\mb\|_2$, it is the OT problem. When $\|\ma\|_2 \neq \|\mb\|_2$, the solution $\hat{\vt}$ is not exist. We define $\y = [\ma, \mb]^{\tranT}$, the UOT problem uses a penalty function for the historgrams: 
\begin{equation}
\label{eq:uot}
\operatorname{UOT}(\ma,\mb) := \min_{\vt \in \R_{+}^{mn}} \vc^{\tranT}t + D_h(\X\vt,\y)
\end{equation}
$D_h$ is the Bregman divergence derived from the norm $h$, $\X = [\mathbf{M}^{\tranT} \mathbf{N}^{\tranT}]^{\tranT}$. 

\subsection{Relationship with Lasso}
The lasso-like problem has a general formula:
$$
\begin{aligned}
f(\vt) = g(\vt) + D_h(\X \vt,\y), t\in \mathbbm{R}^{mn}
\end{aligned}
$$
When $g(\vt) = \lambda \|\vt\|_1$ and $D_h(\X \vt,\y) = \|\X \vt-\y\|_2^2$, this is the $L_2$ regression Lasso problem. It is important to note that $\X$ in UOT is a bit different from the $\X$ in the Lasso problem, the former $\X$ has a specific structure and has only two non-zero elements and is equal to 1, which is quite different to the irregular and dense $\X$ in Lasso problem.


\subsection{Dynamic Screening Framework}

We follow \citep{NEURIPS2021_7b5b23f4}'s framework to introduce the whole dynamic screening technique for the Lasso-like problem:
\begin{equation}
\label{eq:lassolike}
f(\vt) = g(\vt) + d(\X \vt)
\end{equation}

By Frenchel-Rockafellar Duality, we get the dual problem
\begin{thm}
 (Frenchel-Rockafellar Duality) If $d$ and $g$ are proper convex functions on $\mathbbm{R}^{m+n}$ and $\mathbbm{R}^{mn}$. Then we have the following:
 $$
\begin{aligned}
\min_\vt g(\vt) + d(\X\vt) = \max_{\theta} -d^*(-\theta)-g^*(\X^{\tranT}\theta)
\end{aligned}
$$
\end{thm}

Because the primal function $d$ is always convex, the dual function $d^*$ is concave. Assuming $d^*$ is an L-strongly concave problem. we can design an area for any feasible $\tilde{\theta}$ by the strongly concave property:

\begin{thm}\label{circle}
(L-strongly concave) Considering problem \ref{eq:lassolike}, if function $d$ and $g$ are both convex, for $\forall \tilde{\theta} \in{R^{m+n}}$ and satisfied the constraints on the dual problem, we have the following area constructed by its L-strongly concave property:  
$$
\begin{aligned}
\mathcal{R}^{C}:=\theta \in \{\frac{L}{2}\|\theta-\tilde{\theta}\|_2^2+d^*(-\tilde{\theta}) \leq d^*(-\theta)\}
\end{aligned}
$$
\end{thm}
We know that the optimal solution for the dual problem $\hat{\theta}$ satisfied the inequality, so the set is not empty.





