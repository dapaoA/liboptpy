Optimal Transfer (OT) has a long history in mathematics and has recently become prevalent due to its important role in the machine learning community for measuring distances between histograms. It has outperformed traditional methods in many different areas such as domain adaptation \citep{7586038}, generative models \citep{arjovsky2017wasserstein},, graph machine learning \citep{NEURIPS2019_fdd5b16f} and natural language processing. \citep{084adf2f555549c493e0331a00e4ecad} Its popularity is attributed to the introduction of Sinkhorn's algorithm for the entropy optimal transmission problem, \citep{NIPS2013_af21d0c9} which improves the computational speed of the OT problem from $\Theta (n^3)$ of Simplex's method to $\Theta (n^2)$. In order to extend the optimal transmission problem, which can only handle balanced samples, to a wider range of unbalanced samples. The unbalanced optimal transport (UOT) is proposed by modifying the restriction term to a penalty function term. UOT has been used in several applications like computational biology \citep{SCHIEBINGER2019928} , machine learning \citep{DBLP:conf/aistats/JanatiCG19} and deep learning \citep{DBLP:conf/iclr/YangU19}. 

The UOT problem is a regularized version of Kantorovich formulation which replaced the equality constraints with penalty functions on the marginal distributions with a divergence. Many different divergences have been taken into consideration for UOT problems like $KL$ divergence, $l_1$ norm, and $L_2$ norm. When it comes to the solving method, $KL$ penalty with the entropy form can be solved by the Sinkhorn algorithm. It provides the UOT computation with scalability and differentiability but suffers from a larger error of $KL$ divergence and lack of sparsity in solution compared with other regularizers \citep{DBLP:conf/aistats/BlondelSR18}. However, $L_2$ norm could bring a sparse solution, which attracted the attention of researchers and many new algorithms are developed for it. \citep{NEURIPS2021_c3c617a9}, \citep{https://doi.org/10.48550/arxiv.2202.03618} At the same time, The link between the UOT problem with many other well-known problems such as non-negative matrix decomposition and Lasso problem has been discovered, which encourages researchers to improve it by using the rich results in these fields.

Screening is a well-known technique proposed by \citep{ghaoui2010safe} in the field of lasso problems, where the $L_1$ regularizer leads to a sparse solution for the problem. It can pre-select solutions that must be zero theoretically and freeze them before computation. The solutions to many large-scale optimization problems are sparse, and a large amount of computation is wasted on updating the zero elements. With the Safe Screening method, we can identify and freeze the elements that are zero with linear complexity computation before starting the algorithm, thus saving optimization time. the Screening method get attention in recent years and has been improved a lot, New methods such as Dynamic Screening \citep{7128732}, Gap screening method \citep{JMLR:v18:16-577} and Dynamic Sasvi \citep{NEURIPS2021_7b5b23f4}  

The OT and UOT problems produce extremely sparse solutions due to the effectiveness of their optimal transport cost, which is a similar operator to the Lasso problem. We believe that it indicates the potential effectiveness of applying screening technical in the Lasso problem to the UOT problem. Furthermore, Different from the Lasso problem which has a dense constraints matrix, the UOT problem's constraint matrix is extremely sparse and has a unique transport matrix structure, which would benefit the design of screening and the outcome.


\textbf{Contribution}: 
\begin{itemize}
\item We systematically provide the newest framework for the Screening method on the UOT problem. Considering the sparse and specific structure of the UOT problem, we design a new projection method for UOT screening, which hugely improves the screening performance over the general Lasso method.
\item We propose a two-plane screening method for UOT problems, which benefits from UOT's sparse constraints and outperforms the ordinary methods adding only a negligible amount of computation
\end{itemize}



