\section{Notation}
\begin{align}
M &= 
\begin{pmatrix}
\begin{matrix} 1&\\&1 \end{matrix}& & &\dots\dots&\begin{matrix} 1&\\&1 \end{matrix}& &  \\
&\ddots& & \ddots\ddots & & \ddots & & \\
& &\begin{matrix} 1&\\&1 \end{matrix}& \dots\dots& & &\begin{matrix} 1&\\&1 \end{matrix}
\end{pmatrix}\\
N &= 
\begin{pmatrix}
\begin{matrix} 1&1&\dots&1&1\end{matrix} &\dots\dots \\
&\ddots\ddots&\\
&\dots\dots & \begin{matrix} 1&1&\dots&1&1\end{matrix}
\end{pmatrix}
\end{align}

\section{Proof of Theorem 6}
For any $p \in {0,1,...,nm -1}$ we assume that $p = (I,J)$, then we can compute that:
 \begin{equation}
\begin{split} 
\x_p^{\tranT}\tilde{\theta} &= \tilde{\vu}_{I} + \tilde{\vv}_J \\
				       &= {\vu}_{I} + {\vv}_J -  \max_{0\geq j\geq n} \frac{{\vu}_I +{\vv}_j - \lambda \vc_{p}}{2} -  \max_{0 \geq i \geq m} \frac{{\vu}_i +{\vv}_J - \lambda \vc_{p}}{2}\\
				       &= \frac{{\vu}_{I} + {\vv}_J}{2} -  \max_{0\geq j\geq n} \frac{{\vv}_j}{2} -  \max_{0 \geq i \geq m} \frac{{\vu}_i }{2} + \lambda \vc_{p}\\
				       &= \frac{1}{2}\x_p^{\tranT}{\theta}  - \max_{0\geq j\geq n} \frac{{\vv}_j}{2} -  \max_{0 \geq i \geq m} \frac{{\vu}_i }{2}  +\lambda \vc_{p} \\
				       &\leq \lambda \vc_{p} 
 \end{split} 
\end{equation}
As it holds for $\forall p$, $\tilde{\theta} \in \mathcal{R}^{D}$

\section{Proof of Theorem 7}
We Generalize the problem as 
\begin{equation}
\max_{\theta \in \mathcal{R}^{S}_{I}}{ \theta_{I_1} +\theta_{I_2} }
\end{equation}
Considering the center of the circle as $\theta^o$, we define $\theta = \theta^{o} + q$, as ${ \theta^{o}_{I_1} +\theta^{o}_{I_2} }$ is a constant, the problem is equal to  $\max_{\theta \in \mathcal{R}^{S}_{I}}{ q_{I_1} +q_{I_2} }$, we compute the Lagrangrian function of later:
\begin{equation}
\min_{t} \max_{\eta,\mu,\nu > 0}  L(q,\eta,\mu,\nu) =\min_{t}\max_{\eta,\mu,\nu}   {q_{I_1} +q_{I_2}  + \eta( q^{\tranT}q - r)+\mu( a^{\tranT}q - e_a ) + \nu( b^{\tranT}q - e_b )}
\end{equation}

 \begin{equation}
\begin{split} 
\frac{\partial L}{\partial q_i} =  \left\{
\begin{aligned}
1 + 2\eta q_i +\mu a_i + \nu b_i \quad&  i = I_1, I_2\\
 + 2\eta q_i +\mu a_i + \nu b_i \quad&  i \neq I_1, I_2
\end{aligned}
\right.
 \end{split}
\label{eq:lang1}
\end{equation}

 \begin{equation}
\begin{split} 
{q_i}^{*} =  \left\{
\begin{aligned}
-\frac{1+\mu a_i + \nu b_i}{2\eta} \quad&  i = I_1, I_2\\
-\frac{\mu a_i + \nu b_i}{2\eta} \quad&  i \neq I_1, I_2
\end{aligned}
\right.
 \end{split}
\label{eq:lang1}
\end{equation}
We can get the Lagrangrian dual problem:

\begin{equation}
\max_{\eta,\mu,\nu}  L(\eta,\mu,\nu) = \max_{\eta,\mu,\nu>0} -\frac{1+\mu a_{I_1} + \nu b_{I_1}}{2\eta}  -\frac{1+\mu a_{I_2} + \nu b_{I_2}}{2\eta}+ \eta({q^{*}}^{\tranT}q^{*}-r )+\mu( a^{\tranT}q^{*} - e_a ) + \nu( b^{\tranT}q^{*} - e_b )
\end{equation}
Compute the gradient, we can get three equality:
\begin{equation}
\begin{split} 
\frac{\partial L}{\partial \eta} &= \frac{1+\mu a_{I_1} + \nu b_{I_1}}{4\eta^2} +  \frac{1+\mu a_{I_2} + \nu b_ {I_2}}{4\eta^2} + ( {q^{*}}^{\tranT}q^{*} -r) + 2\eta\frac{\partial q^{*}}{\partial \eta}q^{*} +\mu a \frac{\partial q^{*}}{\partial \eta} + \nu b \frac{\partial q^{*}}{\partial \eta}\\
\frac{\partial L}{\partial \mu} &= -\frac{a_{I_1} + a_{I_2}}{2\eta} + 2 \eta q^{*}\frac{\partial q^{*}}{\partial \mu} +( a^{\tranT}q^{*} - e_a ) + \mu a \frac{\partial q^{*}}{\partial \mu} + \nu b \frac{\partial q^{*}}{\partial \mu} \\
\frac{\partial L}{\partial \nu} &= -\frac{b_{I_1} + b_{I_2}}{2\eta} + 2 \eta q^{*}\frac{\partial q^{*}}{\partial \nu} +( b^{\tranT}q^{*} - e_b ) + \nu b \frac{\partial q^{*}}{\partial \nu} + \mu a \frac{\partial q^{*}}{\partial \nu} 
 \end{split}
\end{equation}

As 


