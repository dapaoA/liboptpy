\section{Notation}
\begin{align}
M &= 
\begin{pmatrix}
\begin{matrix} 1&\\&1 \end{matrix}& & &\dots\dots&\begin{matrix} 1&\\&1 \end{matrix}& &  \\
&\ddots& & \ddots\ddots & & \ddots & & \\
& &\begin{matrix} 1&\\&1 \end{matrix}& \dots\dots& & &\begin{matrix} 1&\\&1 \end{matrix}
\end{pmatrix}\\
N &= 
\begin{pmatrix}
\begin{matrix} 1&1&\dots&1&1\end{matrix} &\dots\dots \\
&\ddots\ddots&\\
&\dots\dots & \begin{matrix} 1&1&\dots&1&1\end{matrix}
\end{pmatrix}
\end{align}

\section{Proof of Theorem 6}
For any $p \in {0,1,...,nm -1}$ we assume that $p = (I,J)$, then we can compute that:
 \begin{equation}
\begin{split} 
\x_p^{\tranT}\tilde{\theta} &= \tilde{\vu}_{I} + \tilde{\vv}_J \\
				       &= {\vu}_{I} + {\vv}_J -  \max_{0\geq j\geq n} \frac{{\vu}_I +{\vv}_j - \lambda \vc_{p}}{2} -  \max_{0 \geq i \geq m} \frac{{\vu}_i +{\vv}_J - \lambda \vc_{p}}{2}\\
				       &= \frac{{\vu}_{I} + {\vv}_J}{2} -  \max_{0\geq j\geq n} \frac{{\vv}_j}{2} -  \max_{0 \geq i \geq m} \frac{{\vu}_i }{2} + \lambda \vc_{p}\\
				       &= \frac{1}{2}\x_p^{\tranT}{\theta}  - \max_{0\geq j\geq n} \frac{{\vv}_j}{2} -  \max_{0 \geq i \geq m} \frac{{\vu}_i }{2}  +\lambda \vc_{p} \\
				       &\leq \lambda \vc_{p} 
 \end{split} 
\end{equation}
As it holds for $\forall p$, $\tilde{\theta} \in \mathcal{R}^{D}$

\section{Proof of Theorem 7}
We Generalize the problem as 
\begin{equation}
\max_{\theta \in \mathcal{R}^{S}_{I}}{ \theta_{I_1} +\theta_{I_2} }
\end{equation}
Considering the center of the circle as $\theta^o$, we define $\theta = \theta^{o} + q$, as ${ \theta^{o}_{I_1} +\theta^{o}_{I_2} }$ is a constant, the problem is equal to  $\min_{\theta \in \mathcal{R}^{S}_{I}}{- ( q_{I_1} +q_{I_2} )}$, we compute the Lagrangrian function of later:
\begin{equation}
\min_{q} \max_{\eta,\mu,\nu > 0}  L(q,\eta,\mu,\nu) =\min_{q}\max_{\eta,\mu,\nu}  - {q_{I_1} - q_{I_2}  + \eta( q^{\tranT}q - r^2)+\mu( a^{\tranT}q - e_a ) + \nu( b^{\tranT}q - e_b )}
\end{equation}

 \begin{equation}
\begin{split} 
\frac{\partial L}{\partial q_i} =  \left\{
\begin{aligned}
-1 + 2\eta q_i +\mu a_i + \nu b_i \quad&  i = I_1, I_2\\
  2\eta q_i +\mu a_i + \nu b_i \quad&  i \neq I_1, I_2
\end{aligned}
\right.
 \end{split}
\label{eq:lang1}
\end{equation}

 \begin{equation}
\begin{split} 
{q_i}^{*} =  \left\{
\begin{aligned}
\frac{1- \mu a_i - \nu b_i}{2\eta} \quad&  i = I_1, I_2\\
-\frac{\mu a_i + \nu b_i}{2\eta} \quad&  i \neq I_1, I_2
\end{aligned}
\right.
 \end{split}
\label{eq:lang1}
\end{equation}
We can get the Lagrangrian dual problem:

\begin{equation}
\max_{\eta,\mu,\nu}  L(\eta,\mu,\nu) = \max_{\eta,\mu,\nu>0} \frac{\mu a_{I_1} + \nu b_{I_1}-1}{2\eta}  +\frac{\mu a_{I_2} + \nu b_{I_2}-1}{2\eta}+ \eta({q^{*}}^{\tranT}q^{*}-r^2 )+\mu( a^{\tranT}q^{*} - e_a ) + \nu( b^{\tranT}q^{*} - e_b )
\end{equation}
Compute the gradient, we can get three equality:
\begin{equation}
\begin{split} 
\frac{\partial L}{\partial \eta} &= \frac{1-\mu a_{I_1} - \nu b_{I_1}}{2\eta^2} +  \frac{1-\mu a_{I_2} - \nu b_ {I_2}}{2\eta^2} + ( {q^{*}}^{\tranT}q^{*} -r^2) + 2\eta\frac{\partial q^{*}}{\partial \eta}q^{*} +\mu a \frac{\partial q^{*}}{\partial \eta} + \nu b \frac{\partial q^{*}}{\partial \eta}\\
\frac{\partial L}{\partial \mu} &= \frac{a_{I_1} + a_{I_2}}{2\eta} + 2 \eta q^{*}\frac{\partial q^{*}}{\partial \mu} +( a^{\tranT}q^{*} - e_a ) + \mu a \frac{\partial q^{*}}{\partial \mu} + \nu b \frac{\partial q^{*}}{\partial \mu} \\
\frac{\partial L}{\partial \nu} &= \frac{b_{I_1} + b_{I_2}}{2\eta} + 2 \eta q^{*}\frac{\partial q^{*}}{\partial \nu} +( b^{\tranT}q^{*} - e_b ) + \nu b \frac{\partial q^{*}}{\partial \nu} + \mu a \frac{\partial q^{*}}{\partial \nu} 
 \end{split}
\end{equation}

From KKT we know that, as $\eta,\mu,\nu \neq 0$, we have:

\begin{equation}
\begin{split} 
 {q^{*}}^{\tranT}q^{*} -r^2 &= 0\\
  a^{\tranT}q^{*} - e_a&= 0\\
  b^{\tranT}q^{*} - e_b &=0
 \end{split}
\end{equation}

\begin{equation}
\begin{split} 
  & (1-\mu a_{I_1}-\nu b_{I_1})^2 + (1-\mu a_{I_2}-\nu b_{I_2})^2 + \sum^{m+n}_{i\neq I_1,I_2}(a_i\mu+b_i\nu)^2 - 4\eta^2 r^2 = 0 \\
  & a_{I_1}-\mu a_{I_1}^2-\nu b_{I_1}a_{I_1} + a_{I_2}-\mu a_{I_2}^2-\nu b_{I_2}a_{I_2} - \sum^{m}_{i\neq I_1,I_2}(a_i^2\mu +b_i a_i\nu) - 2\eta {e_a} = 0 \\
  & b_{I_1}-\nu b_{I_1}^2-\mu b_{I_1}a_{I_1} + b_{I_2}-\nu b_{I_2}^2-\mu b_{I_2}a_{I_2} - \sum^{m}_{i\neq I_1,I_2}(b_i^2\nu +b_i a_i\mu) - 2\eta {e_b} = 0 
 \end{split}
\end{equation}
Rearrage as:
\begin{equation}
\begin{split} 
  & 2-2\mu (a_{I_1}+a_{I_2})-2\nu(b_{I_1}+b_{I_2})+  \|a\|^2\mu^2+\|b\|^2\nu^2+2\mu\nu a^{\tranT}b - 4\eta^2 r^2 = 0 \\
  & (a_{I_1}+ a_{I_1}) -  \|a\|^2\mu + a^{\tranT}b\nu - 2\eta {e_a} = 0 \\
  & (b_{I_1}+ b_{I_2}) - \|b\|^2\nu +a^{\tranT}b \mu - 2\eta {e_b} = 0 
 \end{split}
\end{equation}

we have 
\begin{equation}
\begin{split} 
\mu &= \frac{2( e_a\|b\|^2 -  e_ba^{\tranT}b )\eta + (b_{I_1} + b_{I_2}) (a^{\tranT}b) - (a_{I_1}+a_{I_2}) \|b\|^2}{ \|a\|^2 \|b\|^2 -a^{\tranT}b}\\
\nu  &=\frac{2( e_b\|a\|^2 -  e_aa^{\tranT}b )\eta + (a_{I_1} + a_{I_2}) (a^{\tranT}b) - (a_{I_1}+a_{I_2}) \|a\|^2}{ \|a\|^2 \|b\|^2 -a^{\tranT}b}\\
 \end{split}
\end{equation}
set it as:
\begin{equation}
\begin{split} 
\mu &= s_1 \eta + s_2\\ 
\nu  &= u_1 \eta + u_2
 \end{split}
 \label{eq:final}
\end{equation}

Then we can solve the $\eta$ as a quadratic equation:
\begin{equation}
\begin{split} 
0&=a\eta^2+b\eta+c\\
 a&= 4r^2 - s_1^2\|a\|^2 - u_1^2\|b\|^2 -2s_1 u_1a^{\tranT}b\\
b&=2(a_{I_1} + a_{I_2})s_1 +2(b_{I_1} + b_{I_2})u_1 - 2s_1s_2 \|a\|^2 - 2u_1u_2\|b\|^2 - 2(s_1u_2+s_2u_1)a^{\tranT}b  \\
 c&=2(a_{I_1} + a_{I_2})s_2 +2(b_{I_1} + b_{I_2})u_2 -s_2^2\|a\|_1 -u_2^2\|b\|_1 - 2s_2u_2a^{\tranT}b -2
 \end{split}
\end{equation}

Then we can put it back into \ref{eq:final} and get $\mu, \nu$
