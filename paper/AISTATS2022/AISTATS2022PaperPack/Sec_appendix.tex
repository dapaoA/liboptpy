\section{NOTATIONS}
\begin{align}
M &= 
\begin{pmatrix}
\begin{matrix} 1&\\&1 \end{matrix}& & &\dots\dots&\begin{matrix} 1&\\&1 \end{matrix}& & \\
&\ddots& & \ddots\ddots & & \ddots & & \\
& &\begin{matrix} 1&\\&1 \end{matrix}& \dots\dots& & &\begin{matrix} 1&\\&1 \end{matrix}
\end{pmatrix}\\
N &= 
\begin{pmatrix}
\begin{matrix} 1&1&\dots&1&1\end{matrix} &\dots\dots \\
&\ddots\ddots&\\
&\dots\dots & \begin{matrix} 1&1&\dots&1&1\end{matrix}
\end{pmatrix}
\end{align}
\section{PROOFS}
\subsection{Proof of Theorem \ref{eq:uotproj}}
For any $p \in {0,1,...,nm -1}$ we assume that $p = (I,J)$, then we can compute that:
 \begin{equation}
\begin{split} 
\vec{x}_p^{\tranT}\tilde{\theta} &= \tilde{\vec{u}}_{I} + \tilde{\vec{v}}_J \\
				    &= {\vec{u}}_{I} + {\vec{v}}_J - \max_{0\geq j\geq n} \frac{{\vec{u}}_I +{\vec{v}}_j - \lambda \vec{c}_{p}}{2} - \max_{0 \geq i \geq m} \frac{{\vec{u}}_i +{\vec{v}}_J - \lambda \vec{c}_{p}}{2}\\
				    &= \frac{{\vec{u}}_{I} + {\vec{v}}_J}{2} - \max_{0\geq j\geq n} \frac{{\vec{v}}_j}{2} - \max_{0 \geq i \geq m} \frac{{\vec{u}}_i }{2} + \lambda \vec{c}_{p}\\
				    &= \frac{1}{2}\vec{x}_p^{\tranT}{\theta} - \max_{0\geq j\geq n} \frac{{\vec{v}}_j}{2} - \max_{0 \geq i \geq m} \frac{{\vec{u}}_i }{2} +\lambda \vec{c}_{p} \\
				    &\leq \lambda \vec{c}_{p} 
 \end{split} 
\end{equation}
For $\forall p$, we have $\tilde{\theta} \in \mathcal{R}^{D}$

\subsection{Proof of Theorem \ref{eq:divide}}
We Generalize the problem as 
\begin{equation}
\max_{\theta \in \mathcal{R}^{S}_{I}}{ \theta_{I_1} +\theta_{I_2} }
\end{equation}
Considering the center of the circle as $\theta^o$, we define $\theta = \theta^{o} + q$, as ${ \theta^{o}_{I_1} +\theta^{o}_{I_2} }$ is a constant, the problem is equal to $\min_{\theta \in \mathcal{R}^{S}_{I}}{- ( \vec{q}_{I_1} +\vec{q}_{I_2} )}$, we compute the Lagrangrian function of later:
\begin{equation}
\min_{\vec{q}} \max_{\eta,\mu,\nu \geq 0} L(\vec{q},\eta,\mu,\nu) =\min_{\vec{q}}\max_{\eta,\mu,\nu\geq0} - {\vec{q}_{I_1} - \vec{q}_{I_2} + \eta( \vec{q}^{\tranT}\vec{q} - r^2)+\mu( a^{\tranT}\vec{q} - e_a ) + \nu( b^{\tranT}\vec{q} - e_b )}
\end{equation}

 \begin{equation}
\begin{split} 
\frac{\partial L}{\partial \vec{q}_i} = \left\{
\begin{aligned}
-1 + 2\eta \vec{q}_i +\mu a_i + \nu b_i \quad& i = I_1, I_2\\
 2\eta \vec{q}_i +\mu a_i + \nu b_i \quad& i \neq I_1, I_2
\end{aligned}
\right.
 \end{split}
\label{eq:lang1}
\end{equation}

 \begin{equation}
\begin{split} 
{\vec{q}_i}^{*} = \left\{
\begin{aligned}
\frac{1- \mu a_i - \nu b_i}{2\eta} \quad& i = I_1, I_2\\
-\frac{\mu a_i + \nu b_i}{2\eta} \quad& i \neq I_1, I_2
\end{aligned}
\right.
 \end{split}
\label{eq:lang1}
\end{equation}

We can get the Lagrangian dual problem:
\begin{equation}
\max_{\eta,\mu,\nu\geq0} L(\eta,\mu,\nu) = \max_{\eta,\mu,\nu\geq0} \frac{\mu a_{I_1} + \nu b_{I_1}-1}{2\eta} +\frac{\mu a_{I_2} + \nu b_{I_2}-1}{2\eta}+ \eta({\vec{q}^{*}}^{\tranT}\vec{q}^{*}-r^2 )+\mu( a^{\tranT}\vec{q}^{*} - e_a ) + \nu( b^{\tranT}\vec{q}^{*} - e_b )
\end{equation}
From the KKT optimum condition, we know that if
\begin{equation}
\begin{split} 
 \eta ({\vec{q}^{*}}^{\tranT}\vec{q}^{*} -r^2) &= 0\\
 \mu( a^{\tranT}\vec{q}^{*} - e_a)&= 0\\
 \nu(b^{\tranT}\vec{q}^{*} - e_b) &=0
 \end{split}
\end{equation}
We set $\eta^{*}, \mu^{*}, \nu^{*}$ as the solution of the equations, which is also the solution of the dual problem. Firstly, we assume that $\eta^{*}, \mu^{*}, \nu^{*} \neq 0$, then the solution is equal to compute the following equations:

\begin{equation}
\begin{split} 
 & (1-\mu a_{I_1}-\nu b_{I_1})^2 + (1-\mu a_{I_2}-\nu b_{I_2})^2 + \sum^{m+n}_{i\neq I_1,I_2}(a_i\mu+b_i\nu)^2 - 4\eta^2 r^2 = 0 \\
 & a_{I_1}-\mu a_{I_1}^2-\nu b_{I_1}a_{I_1} + a_{I_2}-\mu a_{I_2}^2-\nu b_{I_2}a_{I_2} - \sum^{m}_{i\neq I_1,I_2}(a_i^2\mu +b_i a_i\nu) - 2\eta {e_a} = 0 \\
 & b_{I_1}-\nu b_{I_1}^2-\mu b_{I_1}a_{I_1} + b_{I_2}-\nu b_{I_2}^2-\mu b_{I_2}a_{I_2} - \sum^{m}_{i\neq I_1,I_2}(b_i^2\nu +b_i a_i\mu) - 2\eta {e_b} = 0 
 \end{split}
\end{equation}
Rearranged as:
\begin{equation}
\begin{split} 
 & 2-2\mu (a_{I_1}+a_{I_2})-2\nu(b_{I_1}+b_{I_2})+ \|a\|^2\mu^2+\|b\|^2\nu^2+2\mu\nu a^{\tranT}b - 4\eta^2 r^2 = 0 \\
 & (a_{I_1}+ a_{I_1}) - \|a\|^2\mu - a^{\tranT}b\nu - 2\eta {e_a} = 0 \\
 & (b_{I_1}+ b_{I_2}) - \|b\|^2\nu - a^{\tranT}b \mu - 2\eta {e_b} = 0 
 \end{split}
\end{equation}

we have 
\begin{equation}
\begin{split} 
\mu &= \frac{2( e_ba^{\tranT}b - e_a\|b\|^2 )\eta + (a_{I_1}+a_{I_2}) \|b\|^2 - (b_{I_1} + b_{I_2}) (a^{\tranT}b)}{ \|a\|^2 \|b\|^2 -a^{\tranT}b}\\
\nu &=\frac{2( e_aa^{\tranT}b - e_b\|a\|^2 )\eta + (b_{I_1}+b_{I_2}) \|a\|^2 - (a_{I_1} + a_{I_2}) (a^{\tranT}b)}{ \|a\|^2 \|b\|^2 -a^{\tranT}b}\\
 \end{split}
\end{equation}
set it as:
\begin{equation}
\begin{split} 
\mu &= s_1 \eta + s_2\\ 
\nu &= u_1 \eta + u_2
 \end{split}
 \label{eq:final}
\end{equation}

Then we can solve the $\eta$ as a quadratic equation:
\begin{equation}
\begin{split} 
0&=a\eta^2+b\eta+c\\
 a&= 4r^2 - s_1^2\|a\|^2 - u_1^2\|b\|^2 -2s_1 u_1a^{\tranT}b\\
b&=2(a_{I_1} + a_{I_2})s_1 +2(b_{I_1} + b_{I_2})u_1 - 2s_1s_2 \|a\|^2 - 2u_1u_2\|b\|^2 - 2(s_1u_2+s_2u_1)a^{\tranT}b \\
 c&=2(a_{I_1} + a_{I_2})s_2 +2(b_{I_1} + b_{I_2})u_2 -s_2^2\|a\|_1 -u_2^2\|b\|_1 - 2s_2u_2a^{\tranT}b -2
 \end{split}
\end{equation}

Then we can put it back into \ref{eq:final} and get $\mu, \nu$.

If the solution satisfied the constraints $\eta^{*}, \mu^{*}, \nu^{*} > 0$, then it is the solution.
However, if one of the dual variables is less than 0, the problem would degenerate into a simpler question. 

If only $\eta^{*}$ is larger than 0, 
 $\min_{\theta \in \mathcal{R}^{S}_{I}}{- ( \vec{q}_{I_1} +\vec{q}_{I_2} )} = -\sqrt{2}r$

If only $\mu^{*}$ or $\nu^{*}$ is less than 0, we are optimizing on a sphere cap, the solution can be found in \cite[Appendix B]{NEURIPS2021_7b5b23f4}

if only $\eta^{*} \leq 0$:
As the sphere is inactivated, the problem gets maximum at every point of the intersection of two planes.
\begin{equation}
\min_{\vec{q}} \max_{\mu,\nu \geq 0} L(\vec{q},\mu,\nu) =\min_{\vec{q}}\max_{\mu,\nu\geq0} - {\vec{q}_{I_1} - \vec{q}_{I_2} +\mu( a^{\tranT}\vec{q} - e_a ) + \nu( b^{\tranT}\vec{q} - e_b )}
\end{equation}
To have a solution, the equations satisfied
 \begin{equation}
\begin{split} 
\frac{\partial L}{\partial q} = \left\{
\begin{aligned}
-1+\mu a_i + \nu b_i =0 \quad& i = I_1, I_2\\
-\mu a_i -\nu b_i \quad =0& i \neq I_1, I_2
\end{aligned}
\right.
 \end{split}
\end{equation}

As the equation satisfied, we can just set $\vec{q}_i^{*} = 0, i \neq I_1,I_2$, then we compute the  
 \begin{equation}
 \min_{\theta \in \mathcal{R}^{S}_{I}}{- ( \vec{q}_{I_1} +\vec{q}_{I_2} )} = \frac{a_{I_2}e_b - b_{I_2}e_a -a_{I_1}e_b +b_{I_1}e_a}{a_{I_1}b_{I_2}-a_{I_2}b_{I_1}}
\end{equation}